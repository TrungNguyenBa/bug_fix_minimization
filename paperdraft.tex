\section{Motivation}
	Commit changes processing is an important part of software engineering research, especially in software testing and debugging. Unfortunately, commits are rarely minimized for one specific purpose. This makes it hard for researcher to process the changes since it is often the case that only a few aspects of the changes are interested. In order to bypass this difficulty, most studies apply some assumptions to attempt to filter the unwanted commits. One of the most common assumptions is restriction on the number of lines changed and/or modified. However, these assumptions may not only limit the applications of the proposed methods of tools, but it could also not able to eliminate the commit minimization problem. This could cause significant impacts on the evaluations and results of these studies.
	In this paper, we want to examine the effects of the assumptions mentioned above. We perform the examination by comparing the results using the commits obtained similarly to the original method without minimization and  the results using the same commits but also manually minimized by us. 
 
 \section{Related Work}
Surprisingly, there are not many studies have been done in the past to examining the effects of commit changes minimization. However, the importance of minimizing commits was noticed and addressed in some previous works. Just and ... have tried to ....







\section{References}
